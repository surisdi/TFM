\chapter{On the project organization}
\label{chapter:project_organization}

\paragraph{Important Note:} The current project is considered a \emph{basic research} project, meaning that its goal is not to come up with a final product or patent, but to provide basic and fundamental insights into the computer vision field. Also, as it usually happens with research projects, its final goal was not clearly defined and the intermediate tasks and milestones were unknown at the beginning, being defined as the project was taking shape. For this reason, we consider it is pointless to include a \emph{budget} or a \emph{work plan} section. The \emph{environment impact} section has also been omitted.

\section{Project Development}
\label{section:development}
This project has been developed using Python, and specifically the deep learning library PyTorch \cite{paszke2017pytorch}, which allows computing tensor operations with \ac{GPU} support. The used PyTorch version is the 0.4
The simulations were done on GPUs\dots
To create the dataset, we used Blender \cite{blender}.

The code for all the simulations, as well as the code to create dataset, and to create the plots and figures, can be found in this
\href{www.linktoproject.com}{\underline{link}}\footnote{URL of the project: \url{www.linktoproject.com}.}

\section{Statement of Purpose}
\label{section:sop}
This being a basic research project, its main goal is to increase the general knowledge about a topic, not focusing on solving a specific product-driven problem. It does not have as an objective to improve the state of the art in any specific challenge or benchmark, there is not any value we want to beat. Rather, it aims to better understand the tools we use, to make an analysis of the situation and propose different ways of dealing with it..

Dir que tot i que les bases psicologiques i filosofiques per a aquest
projecte son importants, no son els nostres objectius, que son mes
tecnics: - Trobar manera de evaluar si un concepte s’ha apres - Trobar
manera de fer que la xarxa n’aprengui mes, de conceptes

\section{What part of the project is mine}
\label{section:my_part}
My first contribution to the project is the concept analysis in the ECCV paper \emph{Jointly Discovering Visual Objects and Spoken Words from Raw Sensory Input} \cite{Harwath_2018_ECCV}.
I have done most of my work working very closely with the MIT PhD student Adrià Recasens, who closely supervised my progress. It is somehow difficult to tell what part is mine and what part is not, or to state that some parts are only a result of my work, because the supervision by Adrià and also by Antonio have always been present. In any case, if there is some idea that I explain in this thesis in which I have almost not been involved in (but it is important for the story of the project and to understand my posterior contributions), I will clearly state so.







